%chapter introduce un nuevo cap�tulo

\chapter{Objetivos TODO ESQUEMATIZADO }

%gmmgfngfngfngfngfngfnhgf
%\begin{itemize}
%\item \textbf{ejemplo de lista de puntos}.

%Cocoloco loco loco loco tacacacaca
%\item \textbf{ejemplo de lista de puntos}.\\Cocoloco loco loco loco %tacacacaca
%\item \textbf{ejemplo2 de lista}.
%\end{itemize}

\section{Desarrollar una plataforma rob�tica humanoide}

TODO El primer objetivo de este proyecto es el desarrollo de una plataforma rob�tica humanoide de prop�sito general. Para llegar a ello ser� necesario estudiar los componentes que forman un robot humanoide Sus capacidades deben ser al menos suficientes para desarrolar sobre la plataforma los objetivos del segundo punto.

- Que la plataforma sea lo suficientemente cojonuda para poder hacer esto y mas 

\subsection{Estudiar los componentes que necesita un robot humanoide}

- Se realizar� un estudio de elementos necesarios para un robot
- Se evaluar� cuales funcionan mejor y peor
- Se analizar� cuales son dignos de meterse en mi robot
- Se escoger� una configuraci�n completa 

\subsection{Integrar una c�mara USB}

- Se requiere integrar una c�mara para hacer algoritmos de visi�n
- Se requiere que la c�mara entre integrada f�sicamente en el robot.
- Que la c�mara pueda moverse en el robot.

\subsection{Integrar un controlador}

- Un controlador que permita mover el robot y procesar visi�n.
- Que pueda programarse libremente sin casarse con Robotis.
- Que permita a�adir sensores y actuadores libremente.
- Que se integre fisicamente dentro del robot.
- Que funcione (capacidades de procesamiento, autonom�a... etc)


\subsection{Realizar las modificaciones esctructurales que sean pertinentes}

- Dise�ar piezas que requiera el robot (para montar sensores, c�mara...)
- Dise�ar piezas que no se requieran pero mejoren el comportamiento
- Dise�arlas para que puedan imprimirse con una impresora 3d.
- Imprimirlas todas

\section{Puesta en marcha y programaci�n}

El robot debe ponerse en marcha y funcionar, desde su montaje hasta su participaci�n en CEABOT

\subsection{Desarrollar locomoci�n}

- Controlar movimientos de las articulaciones
- Programar movimientos complejos
- Hacer que ande
Hacer el control de movimientos desde el movimiento de un servo has el movimiento de extremidades y con ello 

\subsection{Desarrollar algoritmos de visi�n}

- Hacer algoritmos de visi�n con OpenCV
- Que funcionen dentro del robot
- Optimizados para funcionar en condiciones



\subsection{Desarrollar aplicaciones de competici�n}

- Apoyandose en vision
- Con los movimientos creados
- Integrando la informaci�n de sensores
- Consiguiendo pasar las pruebas

sando de base las dos cosas anteriores el robot tiene que poder funcionar los suficientemente bien como para hacer programillas para las pruebas concretas.





\paragraph{Palabras clave:} TODO palabraclave1, palabraclave2, palabraclave3.
