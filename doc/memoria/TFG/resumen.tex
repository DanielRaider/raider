%chapter introduce un nuevo cap�tulo
\chapter{Resumen}




En este proyecto se ha desarrollado la plataforma rob�tica mini-humanoide Raider, (Robot Antropom�rfico para la Investigaci�n y Desarrollo en Entornos Reales) con capacidad para actuar de forma aut�noma bas�ndose en algoritmos de visi�n por computador. Para ello se ha dise�ado una configuraci

 integrado en el robot mini-humanoide un sistema de procesamiento de im�genes formado por una c�mara USB y un controlador desarrollado sobre un ordenador de tama�o reducido, As� como diferentes sensores que apoyen a la parte de vision.( TODO mejorar el t�rmino )
Tras dise�ar, fabricar y montar las nuevas piezas se ha procedido a programar el robot. En la programaci�n de la locomoci�n se presentan los pasos que se han seguido desde el movimiento de una articulaci�n simple hasta la combinacion de estos movimientos para producir movimientos mas complejos como la caminata o el control del equilibrio. Por otra parte, se han estudiado y desarrollado algoritmos de visi�n en los que el robot basar� su comportamiento. Mas espec�ficamente, se han desarrollado t�cnicas de path planning basadas en la b�squeda de trayectorias mediante la detecci�n y esquelitizaci�n del espacio navegable basada en el algoritmo de de ZANG SHUEN. Adicionalmente, se han programado otras funciones como el tracking de una pelota o la lectura de c�digos qr.

El procesamiento de im�genes se ha combinado con la informaci�n recibida por los sensores para dise�ar aplicaciones aptas para la competici�n en CEABOT y otros eventos. Para concluir el proyecto, el robot se ha presentado a la edici�n de 2014 de CEABOT.

El proyecto queda como una plataforma viable sobre la que realizar nuevos proyectos por sus capacidades y robustez.



\paragraph{Palabras clave:} palabraclave1, palabraclave2, palabraclave3.

\chapter{Abstract}

(El resumen en ingles)

\paragraph{Keywords:} keyword1, keyword2, keyword3.